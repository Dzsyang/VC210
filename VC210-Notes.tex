%----------------------------------------------------------------------------------------
%	PACKAGES AND DOCUMENT CONFIGURATIONS
%----------------------------------------------------------------------------------------

\documentclass[a4paper,12pt]{article}
\usepackage{geometry}
\usepackage{appendix}
\usepackage{amsfonts,amsmath,amssymb}
\usepackage{enumerate}
\usepackage{float}
\usepackage{geometry}
\usepackage{latexsym}
\usepackage{listings}
\usepackage{multicol,multirow,multido}
\usepackage{siunitx} % Provides the \SI{}{} and \si{} command for typesetting SI units
\usepackage{graphicx} % Required for the inclusion of images
\usepackage{subfigure}
\usepackage{graphicx} % Required for the inclusion of images
\usepackage{subfigure}
\usepackage{multirow}
\usepackage{amsmath} % Required for some math elements 
\usepackage{indentfirst}
\usepackage{times} % Uncomment to use the Times New Roman font
\usepackage{cite}
\usepackage{appendix}
\usepackage{verbatim}
\usepackage[T1]{fontenc}
\usepackage{lastpage}
\usepackage{tabularx}
\usepackage{booktabs}
\usepackage[labelfont=bf,format=plain,justification=centering,singlelinecheck=false]{caption}
\usepackage{fancyhdr} 
\pagestyle{fancy}
\fancyhf{}
\rhead{page \thepage\ of \pageref{LastPage}}
\lhead{VC210 Zixiang Lin}
%\usepackage{float}

%----------------------------------------------------------------------------------------
%	DOCUMENT INFORMATION
%----------------------------------------------------------------------------------------
\begin{document}

\title{\huge VC210 Notes} % Title
\author{Zixiang Lin}

\maketitle % Insert the title, author and date
\thispagestyle{fancy}

\begin{center}

\large Fall 2022\par
\large Prof. Milias Liu\par
\large UM-SJTU Joint Institute
\end{center}

\newpage

\tableofcontents

\newpage



\newpage
%----------------------------------------------------------------------------------------
%	Section 1
%----------------------------------------------------------------------------------------
\section{Foundamentals}

\newpage
%----------------------------------------------------------------------------------------
%	Section 2
%----------------------------------------------------------------------------------------
\section{Atomic Theory}

\newpage
%----------------------------------------------------------------------------------------
%	Section 3
%----------------------------------------------------------------------------------------
\section{Basic Quantum Mechanics}

\newpage
%----------------------------------------------------------------------------------------
%	Section 4
%----------------------------------------------------------------------------------------
\section{Chemical Bonds}

\newpage
%----------------------------------------------------------------------------------------
%	Section 5
%----------------------------------------------------------------------------------------
\section{Gases}
\subsection{The Nature of Gases} Eleven elements are gases under normal conditions: H, He, N, O, F, Ne, Cl, Ar, Kr, Xe, Rn. Low molar mass compounds such as carbon dioxide, hydrogen chlorideare also gases.
\subsection{The Unit of Pressure}
$1 Pa=1 kg\cdot m^{-1}\cdot s^{-2}$. $10^{5} Pa=1 bar$.\par $760 mmHg=760 Torr=1atm=1.013\times 10^{5} Pa$.
\subsection{The Gas Laws}
\textbf{Boyle's Pressure Experiment} $PV=Const.$\par
\textbf{Charles' Law} $V/T=Const.$\par No real gas has zero volume. Lowest possible temperature: -273.15°C.
\subsection{Avogadro‘s Principle}
All gases occupy the same volume under the same conditions of temperature
and pressures. The molar volume of all gases is close to $22.4 L\cdot mol^{-1}$ at 0 °C and 1 atm.
\subsection{Standard Conditions}
\noindent\textbf{Standard Ambient Temperature and Pressure (SATP)}\par SATP means exactly 25 °C (298.15 K) and exactly 1 bar. The molar volume of an ideal gas is $24.79 L\cdot mol^{-1}$.\\
\noindent\textbf{Standard Temperature and Pressure (STP)}\par STP means 0 °C (273.15 K) and 1 atm (both exactly).
The molar volume of an ideal gas is $22.41 L\cdot mol^{-1}$.
\subsection{The Ideal Gas Law}
\begin{center}
$pV=nRT$
\end{center}\par
The constant $R = 8.314 J\cdot K^{-1}\cdot mol^{-1}$.\par
The ideal gas law is a limiting law, valid only as $p\rightarrow0$.\par
The differences between idel gases and real gases are significant at high pressures and low temperatures.
\subsection{Combined Gas Law}
\begin{center}
$\dfrac{p_{1}V_{1}}{n_{1}T_{1}}=\dfrac{p_{2}V_{2}}{n_{2}T_{2}}$
\end{center}
\subsection{Change n for The Ideal Gas Law}
\textbf{Molar Concentration} $M=\dfrac{p}{RT}$.\par
\textbf{Gas Density} $d=\dfrac{Mp}{RT}$.
\subsection{Mixtures of Gases}
A mixture of gases behaves like a single pure gas.\par
\textbf{Dalton's Law of Partial Pressures}\par
Dalton concluded that the total pressure is the sum of the individual pressures of each gas.
\begin{center}
$\chi_{A}=\dfrac{n_{A}}{n_{A}+n_{B}+...}$, $p_{A}=\chi_{A}p_{T}$
\end{center}
\subsection{The Kinetic Model of Gases}
The kinetic model (KMT) of a gas allows us to derive the quantitative relation between pressure and the speeds of the molecules.\par
\textbf{Root Mean Square Speed} $v_{rms}=(\frac{3RT}{M})^{2}$.\par
\textbf{Effusion} In effusion, molecules escape through a small hole in a barrier into a region of low pressure. And we can find $\overline{v}\propto \frac{1}{\sqrt{m}}$.\par
\textbf{The Maxwell Distribution of Speeds} $f(v)=4\pi (\frac{M}{2\pi RT})^{\frac{3}{2}}v^{2}e^{-Mv^{2}/2RT}$.\par
For the same temparature, the greater the molar mass, the lower the speed.\par
For the same molar mass, the higher the temperature, the higher the average speed and the broader the spread of speeds.
\subsection{Deviations from Ideality}
Gases can be condensed to liquids when cooled or compressed.\par
A measurement of the compression factor $Z=\frac{V_{m, real}}{V_{m, ideal}}$.\par
At low pressures the attractive forces are dominant and $Z<1$.\par
At high pressures, repulsive forces become dominant and $Z>1$ for all gases.\par
\textbf{van der Waals equation} 
$(p+a\frac{n^{2}}{V^{2}})(V-nb)=nRT$.\par
Parameter “a” represents the attraction between molecules; the value is large for strongly attracting molecules.
Parameter “b” represents the role of repulsions; it can be thought of as representing the volume.

\newpage
%----------------------------------------------------------------------------------------
%	Section 6
%----------------------------------------------------------------------------------------
\section{Liquids and Solids}
\subsection{Different Types of Intermolecular Forces}
\begin{center}
  \begin{tabular}{ccc}
    \toprule
    Type of interaction & $E_{p}$ dependence & Interacting species \\
    \hline
    ion-ion    & $1/r$  & ions  \\
    ion-dipole  & $1/r^{2}$  & ions and polar molecules  \\
    dipole-dipole (stationary)  & $1/r^{3}$  & stationary polar molecules  \\
    dipole-dipole (rotating)  & $1/r^{6}$  &rotating polar molecules  \\
    hydrogen bonding &   &special case of dipole-dipole: N-H, O-H, F-H  \\
    \bottomrule
  \end{tabular}
\end{center}
\subsection{London Forces}
Attractive forces between nonpolar molecules are London forces.\par
Even nonpolar noble gases can be liquefied, as well as many nonpolar compounds.\par
\textbf{London interaction} $E_{p} \propto \dfrac{\alpha_{1}\alpha_{2}}{r^{6}}$.\par
\textbf{Influence}\par
Size: As size increases (more shells) $\Rightarrow$ polarizability increases $\Rightarrow$ melting and boiling points increase.\par
Shape: Rod-like molecules have a greater surface area, more contact points for molecules to join together (high melting and boiling points); Ball or spherical shaped molecules have fewer contact points for molecules to join together (low melting and boiling points).
\subsection{Hydrogen Bonding}
Very strong interaction between molecules that is specific to molecules with certain types of atoms (the second strongest only to ion-ion interaction).\par
A hydrogen bond is denoted by a dotted line, $X-H...X$, X = N, O, F.
\subsection{Surface Tension}
\textbf{Viscosity} is a liquid’s resistance to flow: the higher the viscosity of the liquid,
the more sluggish the flow. ↑Viscosity indicates, ↑intermolecular strength.\par
\textbf{Surface tension} is the reason that the surface of a liquid is smooth. Strong forces pull the molecules together, with a net inward pull.\par
The upward curved meniscus (concave) of water forms because both water and glass have comparable forces: Adhesion $\approx$ Cohesion.\par
The downward meniscus (convex) of mercury forms because the cohesive forces in mercury is stronger than between mercury atoms and the glass: Cohesion $>$ Adhesion.























\newpage
\end{document}